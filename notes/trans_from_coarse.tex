% !TEX TS-program = pdflatex
% !TEX encoding = UTF-8 Unicode

% This is a simple template for a LaTeX document using the "article" class.
% See "book", "report", "letter" for other types of document.

\documentclass[11pt]{article} % use larger type; default would be 10pt

\usepackage[utf8]{inputenc} % set input encoding (not needed with XeLaTeX)

%%% Examples of Article customizations
% These packages are optional, depending whether you want the features they provide.
% See the LaTeX Companion or other references for full information.

%%% PAGE DIMENSIONS
\usepackage{geometry} % to change the page dimensions
\geometry{a4paper} % or letterpaper (US) or a5paper or....
% \geometry{margin=2in} % for example, change the margins to 2 inches all round
% \geometry{landscape} % set up the page for landscape
%   read geometry.pdf for detailed page layout information

\usepackage{graphicx} % support the \includegraphics command and options

% \usepackage[parfill]{parskip} % Activate to begin paragraphs with an empty line rather than an indent

%%% PACKAGES
\usepackage{booktabs} % for much better looking tables
\usepackage{array} % for better arrays (eg matrices) in maths
\usepackage{paralist} % very flexible & customisable lists (eg. enumerate/itemize, etc.)
\usepackage{verbatim} % adds environment for commenting out blocks of text & for better verbatim
\usepackage{subfig} % make it possible to include more than one captioned figure/table in a single float
% These packages are all incorporated in the memoir class to one degree or another...

%%% HEADERS & FOOTERS
\usepackage{fancyhdr} % This should be set AFTER setting up the page geometry
\pagestyle{fancy} % options: empty , plain , fancy
\renewcommand{\headrulewidth}{0pt} % customise the layout...
\lhead{}\chead{}\rhead{}
\lfoot{}\cfoot{\thepage}\rfoot{}

%%% SECTION TITLE APPEARANCE
\usepackage{sectsty}
\allsectionsfont{\sffamily\mdseries\upshape} % (See the fntguide.pdf for font help)
% (This matches ConTeXt defaults)

%%% ToC (table of contents) APPEARANCE
\usepackage[nottoc,notlof,notlot]{tocbibind} % Put the bibliography in the ToC
\usepackage[titles,subfigure]{tocloft} % Alter the style of the Table of Contents
\renewcommand{\cftsecfont}{\rmfamily\mdseries\upshape}
\renewcommand{\cftsecpagefont}{\rmfamily\mdseries\upshape} % No bold!



%%% This is all Custom
\usepackage{amsmath}
\DeclareMathOperator{\interp}{interp}
\DeclareMathOperator{\QR}{QR}
\DeclareMathOperator{\sd}{sd}
\newcommand{\mat}{\mathbf}

%%% END Article customizations

%%% The "real" document content comes below...

\title{Singular values of covariance matrices under localization}
\author{Travis Harty}
\date{} % Activate to display a given date or no date (if empty),
         % otherwise the current date is printed

\begin{document}
\maketitle

\section{Optimal linear transform}

We update our state $x\sim N(\mu, \mat{P})$ using observation
$y = \mat{H} \mu + \epsilon$, where $\epsilon \sim N(0, \mat{R})$,
$\dim{x} = N_x$, and $\dim{y} = N_y$.
We will perform this update by first transforming our variables using
linear transformations $\mat{T}_x$ and $\mat{T}_y$ for $x$ and $y$ respectively
such that $\mat{T}_x x = \tilde{x} \sim N(\tilde{\mu}, \mat{I})$ and
$\mat{T}_y y = \tilde{y} = \mat{\Sigma} \tilde{\mu} + \tilde{\epsilon}$, where
$\mat{\Sigma}$ is diagonal, and $\tilde{\epsilon} \sim N(0, \mat{I})$.

Take,
\[
  x' = \mat{P}^{-1/2} x
  \text{ and }
  y' = \mat{R}^{-1/2} y.
\]
We then have,
\[
  y = \mat{H} \mu + \epsilon
\]
\[
  \mat{R}^{1/2} y' = \mat{H}
  \mat{P}^{1/2} \mu' + \epsilon
\]
\[
  y' = \mat{R}^{-1/2}\mat{H}
  \mat{P}^{1/2} \mu' + \epsilon',
\]
where $\epsilon \sim N(0, \mat{I})$.
We can then take the singular value decomposition of
$\mat{R}^{-1/2} \mat{H} \mat{P}^{1/2}$ yielding,
\[
  y' = \mat{U} \mat{\Sigma} \mat{V}^{T} \mu' + \epsilon'
\]
\[
  \mat{U}^T y' = \mat{\Sigma} (\mat{V}^T \mu') + \tilde{\epsilon}
\]
\[
  \tilde{y}= \mat{\Sigma} \tilde{\mu} + \tilde{\epsilon}
\]
where $\mat{\Sigma}$ is diagonal and $\tilde{\epsilon} \sim N(0,
\mat{I})$.
We have $\mat{T}_x = \mat{V}^T \mat{P}^{-1/2}$ and
$\mat{T}_y = \mat{U}^T \mat{R}^{-1/2}$.





\section{Dimension reduction optimal linear transform}

We again update our state $x \sim N(\mu, \mat{P})$ with
observations $y = \mat{H} x + \epsilon$, where $\epsilon \sim N(0,
\mat{R})$.
We will now calculate
\[
  \mat{P} = \mat{Q}_x \mat{\Lambda}_x \mat{Q}_x^T
  \text{ and }
  \mat{R} = \mat{Q}_y \mat{\Lambda}_y \mat{Q}_y^T
\]
and choose $N_{\lambda_x}$ and $N_{\lambda_y}$ that are the number of
eigenvalues to keep for $P$ and $R$ respectively such that,
\[
  \dim(\mat{Q}_x) = (N_x, N_{\lambda_x})\text{; }\dim(\mat{\Lambda}_x) =
  (N_{\lambda_x}, N_{\lambda_x})
\]
\[
  \text{ and }
\]
\[
  \dim(\mat{Q}_y) = (N_y, N_{\lambda_y})\text{; }\dim(\mat{\Lambda}_y) =
  (N_{\lambda_y}, N_{\lambda_y})
\]

We can then repeat the above calculations while reducing the
transformed variables' dimensions.
Take,
\[
  x' = \mat{\Lambda}_x^{-1/2} \mat{Q}_x^T x
  \text{ and }
  y' = \mat{\Lambda}_y^{-1/2} \mat{Q}_y^T y
\]
Note that we have,
\[
  x \approx \mat{Q}_x \mat{\Lambda}_x^{1/2} x'
\]
\[
  y \approx \mat{Q}_y \mat{\Lambda}_y^{1/2} y'
\]
where there is equality when $N_{\lambda_x}$ ($N_{\lambda_y}$) is the
true rank of $\mat{P}$ ($\mat{R}$).

Assuming that $N_{\lambda_x}$ and $N_{\lambda_y}$ are the ranks of $P$
and $R$ respectively, we have,
\[
  y = \mat{H} \mu + \epsilon
\]
\[
  \mat{Q}_y \mat{\Lambda}_y^{1/2} y' = \mat{H}
  \mat{Q}_x \mat{\Lambda}_x^{1/2} \mu' + \epsilon
\]
\[
  y' = \mat{\Lambda}_y^{-1/2} \mat{Q}_y^T \mat{H}
  \mat{Q}_x \mat{\Lambda}_x^{1/2} \mu' + \epsilon',
\]
where $\epsilon' \sim N(0, \mat{I}_{N_{\lambda_y}})$.
We can then take the singular value decomposition of
$\mat{\Lambda}_y^{-1/2} \mat{Q}_y^T \mat{H} \mat{Q}_x
\mat{\Lambda}_x^{1/2}$ yielding,
\[
  y' = \mat{U} \mat{\Sigma} \mat{V}^{T} \mu' + \epsilon'
\]
\[
  \mat{U}^T y' = \mat{\Sigma} (\mat{V}^T \mu') + \tilde{\epsilon}
\]
\[
  \tilde{y}= \mat{\Sigma} \tilde{\mu} + \tilde{\epsilon}
\]
where $\mat{\Sigma}$ is diagonal and $\tilde{\epsilon} \sim N(0,
\mat{I}_{N_{\lambda_y}})$.
We have $\mat{T}_x = \mat{V}^T \mat{\Lambda}_x^{-1/2} \mat{Q}_x^T$ and
$\mat{T}_y = \mat{U}^T \mat{\Lambda}_y^{-1/2} \mat{Q}_y^T$.
Note that we have,
\[
  \mat{T}_x \mat{T}_x^{-R} = \mat{V}^T \mat{\Lambda}_x^{-1/2}
  \mat{Q}_x^T \mat{Q}_x \mat{\Lambda}_x^{1/2} \mat{V} = \mat{I}_{N_{\lambda_x}},
\]
but there is no left inverse.
However, the right inverse is something like a left inverse in that
\[
  \mat{T}_x^{-R} \mat{T}_x = \mat{Q}_x \mat{\Lambda}_x^{1/2} \mat{V}
  \mat{V}^T \mat{\Lambda}_x^{-1/2} \mat{Q}_x^T
\]
\[
  \mat{T}_x^{-R} \mat{T}_x = \mat{Q}_x \mat{Q}_x^T
\]
which is equal to the identity matrix only if $P$ is full rank, and
all eigenvectors are kept.
However, if $\mu$ is in the span of the columns of $\mat{Q}_x$, then
we have,
\[
  \mat{T}_x^{-R} \mat{T}_x \mu = \mat{Q}_x \mat{Q}_x^T \mu = \mu
\]
even if $\mat{T}_x$ does not have a left inverse.
Furthermore, if $\mu$ is not in the span of the columns of
$\mat{Q}_x$, this is possible even if the range of $\mat{P}$ and
$\mat{Q}_x$ are equal, then we have,
\[
  (\mat{T}_x^{-R} \mat{T}_x + \mat{I}_{N_x} - \mat{Q}_x \mat{Q}_x^T)
  \mu = \mat{Q}_x \mat{Q}_x^T \mu  + \mu - \mat{Q}_x \mat{Q}_x^T \mu =
  \mu.
\]
We can therefore always recover the parts of $\mu$ that are removed
because of the representation of $\mu$ in the column space of $\mat{Q}_x$.

\section{Multi-scale optimal transform}

Suppose that we have two ensembles representing the state $x$.
One ensemble, $\mat{X}_c$ will have many members, but will have a lower
resolution and only represent large scale structures of the state.
The other ensemble, $\mat{X}$ will have fewer members, but will
have a finer resolution and therefore be able to represent fine scale
structures of the state.
We then have,
\[
  \dim \mat{X}_c = N_{x_c}, N_{e_c}
\]
\[\text{and}\]
\[
  \dim \mat{X} = N_x, N_e
\]
where $N_{x_c} < N_x$ and $N_{e_c} > N_e$.

Taking the singular value decomposition of the ensemble of
coarse perturbations, $\mat{X}_c^*$, will give us the eigenvalues and
eigenvectors of the sample covariance matrix derived from $\mat{X}_c$.
\[
  \mat{X}_c^{*} = \left( \mat{X}_c - \frac{1}{N}\bar{\mat{X}}_c
  \mat{1}_c \right) / \sqrt{N_{e_c} - 1}
\]
\[
  \mat{Q}_x \mat{\Lambda}_x^{1/2} \mat{V}^T_{temp} = \mat{X}_c^*
\]
\[
  \mat{Q}_x \mat{\Lambda}_x \mat{Q}_x^T = (\mat{X}_c^*)
  {(\mat{X}_c^*)}^{T}
\]
where $\mat{1}_c$ is a matrix of all ones of with the same dimension as
$\mat{X}_c$ and $\bar{\mat{X}}_c$ is the sample mean of $\mat{X}_c$.
Assuming that we know $\mat{Q}_y$ and $\mat{\Lambda}_y$, we can then
calculate $\mat{U}_c$ and $\mat{V}_c$,
\[
  \mat{U}_c \mat{\Sigma}_c \mat{V}_c^T = \mat{\Lambda}^{-1/2}_y
  \mat{Q}_y^T \mat{H} \mat{Q}_x \mat{\Lambda}^{1/2}_x.
\]
This then allows us to use

\section{Equivalence to standard KF}

Sticking with the above notation, we can calculate the standard Kalman
filter update and the corresponding equations:
\[
  \mat{K} = \mat{P}\mat{H}^T (\mat{H} \mat{P} \mat{H}^T + \mat{R})^{-1},
\]
\[
  \mu^a = \mu + \mat{K} (y - \mat{H} \mu),
\]
\[
  \mat{P}^a = (\mat{I} - \mat{K} \mat{H}) \mat{P}.
\]
Alternatively, in terms of our transformed equations, we have:
\[
  \tilde{\mat{K}} = \mat{I}_{N_{\lambda_x}} \mat{\Sigma}^T ( \mat{\Sigma} \mat{I}_{N_{\lambda_x}}
  \mat{\Sigma}^T + \mat{I}_{N_{\lambda_y}})^{-1} = \mat{\Sigma}^T (\mat{\Sigma}
  \mat{\Sigma}^T + \mat{I}_{N_{\lambda_y}})^{-1}
\]
\[
  \tilde{\mu}^a = \tilde{\mu} + \tilde{\mat{K}}(\tilde{y} + \mat{\Sigma} \tilde{\mu})
\]
\[
  \tilde{\mat{P}}^a = (\mat{I}_{N_{\lambda_x}} - \tilde{\mat{K}} \mat{\Sigma}) \tilde{\mat{P}}.
\]
We can then convert $\tilde{x}^a$ and $\tilde{\mat{P}}^a$ back to the
original $x$-space through,
\[
  \mu^a_1 = \mat{T}_x^{-R} \tilde{\mu}^a
\]
\[
  \text{ and }
\]
\[
  \mat{P}^a_1 = \mat{T}_x^{-R} \tilde{\mat{P}}^a \left(\mat{T}_x^{-R}\right)^T.
\]

Assuming that the ranks of $\mat{P}$ and $\mat{R}$ are $N_{\lambda_x}$
and $N_{\lambda_y}$ respectively,
\[
  \tilde{\mat{K}} = \mat{\Sigma}^T (\mat{\Sigma} \mat{\Sigma}^T +
  \mat{I}_{N_{\lambda_y}})^{-1}
\]
\[
  \tilde{\mat{K}} = \mat{T}_x \mat{P} \mat{T}_x^T (\mat{T}_y \mat{H}
  \mat{T}_x^{-R})^T \left((\mat{T}_y \mat{H} \mat{T}_x^{-R})(\mat{T}_y
  \mat{H} \mat{T}_x^{-R})^T + \mat{T}_y \mat{R} \mat{T}_y^T \right)^{-1}
\]
\[
  \tilde{\mat{K}} = \mat{T}_x \mat{P} \mat{T}_x^T \left( \mat{T}_x^{-R}
  \right)^T \mat{H}^T \mat{T}_y^T
  \left( \mat{T}_y \mat{H} \mat{T}_x^{-R} \left( \mat{T}_x^{-R}
    \right)^T \mat{H}^T \mat{T}_y^T + \mat{T}_y \mat{R} \mat{T}_y^T \right)^{-1}
\]
\[
  \tilde{\mat{K}} = \mat{T}_x \mat{P} \mat{Q}_x \mat{Q}_x^T \mat{H}^T \mat{T}_y^T
  \left( \mat{T}_y \left( \mat{H} \mat{Q}_x \Lambda_x \mat{Q}_x^T
      \mat{H}^T + \mat{R}\right) \mat{T}_y^T \right)^{-1}
\]
If we assume that $\mu$ is in the span of $\mat{Q}_x$, we have,
\[
   x^a_1 = \mat{T}_x^{-R} \tilde{x}^a
\]
\[
   x^a_1 = \mat{T}_x^{-R} \left( \tilde{x} + \tilde{K} \right)
\]


% \section{Transformation from coarse}



% We have two ensembles representing the same system.
% The ensemble $X$ is a coarse representation of the state, but has many ensemble members.
% The ensemble $Z$ is a fine representation of the sates, but has few ensemble members.
% \[
% \dim{X} = N_x, N_{ex}
% \]
% \[
% \dim{Z} = N_z, N_{ez}
% \]

% In order to find the eigenvalues and eigenvectors of the sample correlation of $X$, we take the singular value decomposition of $\tilde{X} = {(N_{ex} - 1)}^{-1/2} (X - \bar{X})/ \sd(X)$, where $\sd(X)$ is the standard deviation of each element of $X$.
% \[
% U_x S_x V_x^T = \tilde{X}
% \]
% This means that,
% \[
% U_x S_z^2 U_x^T = \hat{C}_x
% \]
% where $\hat{C}_x$ is the sample correlation of $X$.

% We want to then use $U_x$, or its leading columns, to estimate the leading $U_z$.
% To do this, we interpolate $U_x$ to the $z$ space and then use QR factorization to ensure the interpolated $u_x$'s are orthonormal:
% \[
% U_{x} = \interp (U_x)
% \]
% \[
% U_{xi}, R = \QR (U_{xi}).
% \]
% We then must choose how many of the columns of $U_{xi}$ should be used.
% The best way to make this choice is unclear.
% A few ways that I am considering are:
% \begin{enumerate}
% \item Keep $U_{xi}$ based on the cumulative sum of their corresponding eigenvalues.
% \item Keep $U_{xi}$ based on the rate of change of their corresponding eigenvalues.
% \item Keep $U_{xi}$ based on some measure of the length scale produced by the low rank approximation of $\hat{C}_x$ and the residual.
% \item Keep $U_{xi}$ based on how orthogonal the interpolated vector is in $z$ space before orthogonalization.
% \end{enumerate}
% It should be noted that keeping all columns can be detrimental to the assimilation process.
% If some of the columns of $U_{xi}$ are representing small scale structures, then in the following steps they will still be used to represent some of these small scale structures.

% After choosing which columns of $U_{xi}$ to keep, we must determine what eigenvalues they should have.
% We do this by taking $\lambda = u_{xi}^T \hat{C}_z u_{xi}$ as the eigenvalue of $u{xi}$ for our approximation of $C_z$.
% Alternatively, we can take $\lambda = {(\tilde{Z}^T u_{xi})}^T (\tilde{Z}^T u_{xi})$ where $\tilde{Z}$ is defined similarly as $\tilde{X}$.
% This will give us the leading eigenvectors and an approximation of their eigenvalues of $C_Z$.

% These eigenvectors and eigenvalues represent the large scale structure of the problem.
% To then find the small scales of the problem, we take
% \[
% C_z^\bot = C_z^\parallel - \hat{C}_z
% \]
% where
% \[
% C_z^\parallel = U_{xi} \Lambda_{xi}  U_{xi}^T
% \]
% and $\Lambda_{xi}$ is the diagonal matrix with the $\lambda$s described above on the diagonal.
% We can then localize $C_z^\bot$ to get at the small scales that are represented in $Z$.

% We must choose how to localize $C_z^\bot$.
% One reasonable expectation is that the scales in $C_z^\bot$ will be shorter or similar to $N_z/N_x$.
% This choice will also be affected by the choice of how many columns of $U_{xi}$ to keep.
% Once the localization matrix $L$ is chosen, we can then generate our localized correlation matrix as:
% \[
% C_z^{loc} = C_z^\parallel + L \circ C_Z^\bot
% \]
% and the corresponding covariance matrix as:
% \[
% P_z^{loc} = D_z C_z^{loc} D_z.
% \]
% where $D_z$ is the diagonal matrix with sample standard deviations of $Z$ on the diagonal.

% We can then find the leading eigenvectors and eigenvalues of $C_z^{loc}$:
% \[
% Q \Lambda Q^T = C_z^{loc}
% \]
% and use them to transform the $z$ variable.
% First, we must calculate a whitening transformation
% \[
% T_w = \Lambda^{-1/2} Q^T
% \]
% and its right inverse
% \[
% T_w^i = Q \Lambda^{1/2}
% \]
% and the singular value decomposition
% \[
% U \Sigma V^T = R^{-1/2} H T_w^i.
% \]
% We can then define transformations for both $z$ and $y$
% \[
% T_z = V^T T_w
% \]
% \[
% T_y = U^T R^{-1/2}
% \]

% We then know $z^* = T_z z$ and $y^* = T_y y$ have identity covariance matrices and
% \[
% y^* = \Sigma z^* + \epsilon^*
% \]
% \[
% U^T R^{-1/2} y = \Sigma V^T \Lambda^{-1/2} Q^T z + \epsilon^*
% \]
% \[
% y = R^{1/2} U \Sigma V^T \Lambda^{-1/2} Q^T z  + \epsilon
% \]
% \[
% y = R^{1/2}R^{-1/2} H Q \Lambda^{1/2} \Lambda^{-1/2} Q^T z  + \epsilon
% \]
% \[
% y = H z  + \epsilon
% \]
% The question is: can I do this better?
% Specifically:
% \begin{enumerate}
% \item Can I calculate the eigenvectors and values of $C^{loc}$ without generating $C^\parallel$ or $C^\bot$ explicitly?
% \item Can I calculate the eigenvectors and values of $P$ from those of $C$ without generating $C$?
% \item Can I calculate the singular vectors and values of $R^{-1/2} H T_w^i$ from the singular value decomposition of $R$, $H$, and $T_w^i$?
% \end{enumerate}




















\end{document}

%%% Local Variables:
%%% mode: latex
%%% TeX-master: t
%%% End:
