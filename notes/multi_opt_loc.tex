% !TEX TS-program = pdflatex
% !TEX encoding = UTF-8 Unicode

% This is a simple template for a LaTeX document using the "article" class.
% See "book", "report", "letter" for other types of document.

\documentclass[11pt]{article} % use larger type; default would be 10pt

\usepackage[utf8]{inputenc} % set input encoding (not needed with XeLaTeX)

%%% Examples of Article customizations
% These packages are optional, depending whether you want the features they provide.
% See the LaTeX Companion or other references for full information.

%%% PAGE DIMENSIONS
\usepackage{geometry} % to change the page dimensions
\geometry{a4paper} % or letterpaper (US) or a5paper or....
% \geometry{margin=2in} % for example, change the margins to 2 inches all round
% \geometry{landscape} % set up the page for landscape
%   read geometry.pdf for detailed page layout information

\usepackage{graphicx} % support the \includegraphics command and options

% \usepackage[parfill]{parskip} % Activate to begin paragraphs with an empty line rather than an indent

%%% PACKAGES
\usepackage{booktabs} % for much better looking tables
\usepackage{array} % for better arrays (eg matrices) in maths
\usepackage{paralist} % very flexible & customisable lists (eg. enumerate/itemize, etc.)
\usepackage{verbatim} % adds environment for commenting out blocks of text & for better verbatim
\usepackage{subfig} % make it possible to include more than one captioned figure/table in a single float
% These packages are all incorporated in the memoir class to one degree or another...

%%% HEADERS & FOOTERS
\usepackage{fancyhdr} % This should be set AFTER setting up the page geometry
\pagestyle{fancy} % options: empty , plain , fancy
\renewcommand{\headrulewidth}{0pt} % customise the layout...
\lhead{}\chead{}\rhead{}
\lfoot{}\cfoot{\thepage}\rfoot{}

%%% SECTION TITLE APPEARANCE
\usepackage{sectsty}
\allsectionsfont{\sffamily\mdseries\upshape} % (See the fntguide.pdf for font help)
% (This matches ConTeXt defaults)

%%% ToC (table of contents) APPEARANCE
\usepackage[nottoc,notlof,notlot]{tocbibind} % Put the bibliography in the ToC
\usepackage[titles,subfigure]{tocloft} % Alter the style of the Table of Contents
\renewcommand{\cftsecfont}{\rmfamily\mdseries\upshape}
\renewcommand{\cftsecpagefont}{\rmfamily\mdseries\upshape} % No bold!



%%% This is all Custom
\usepackage{amsmath}
\DeclareMathOperator{\interp}{interp}
\DeclareMathOperator{\QR}{QR}
\DeclareMathOperator{\sd}{sd}
\newcommand{\mat}{\mathbf}

%%% END Article customizations

%%% The "real" document content comes below...

\title{Optimal linear transform for multi-scale localization}
\author{Travis Harty}
\date{} % Activate to display a given date or no date (if empty),
         % otherwise the current date is printed

\begin{document}
\maketitle

We update our state $x\sim N(\mu, \mat{P})$ using observation
$y = \mat{H} \mu + \varepsilon$, where $\varepsilon \sim N(0, \mat{R})$,
$\dim{x} = N_x$, and $\dim{y} = N_y$.
We will look for transformations $\mat{T}_{xl}$, $\mat{T}_{yl}$,
$\mat{T}_{xs}$, and $\mat{T}_{ys}$ such that $x_{l} = \mat{T}_{xl} x$
and $y_{l} = \mat{T}_{yl} y$ contain long-scale information and $x_s =
\mat{T}_{xs} x$ and $y_s = \mat{T}_{ys} y$ contains short-scale
information.

We begin by calculating the eigenvalue decomposition of $\mat{P}$ and
$\mat{R}$,
\[
  \mat{P} = \mat{Q}_x \mat{\Lambda}_x \mat{Q}_x^T
  \text{ and }
  \mat{R} = \mat{Q}_y \mat{\Lambda}_y \mat{Q}_y^T.
\]
We do not assume that $\mat{P}$ or $\mat{R}$ are full rank and that
the decompositions are in reduced form.

We can now reform the relationship between $x$ and $y$.
First we must notice that even if the image of $\mat{P}$ is equal to
the image of $\mat{Q}_x$, this does not mean that $\mu$ is in the
image of $\mat{Q}_x$.
This means that if we project $\mu$ onto the preimage of $\mat{Q}_x$
we will lose the part of $\mu$ that is in the kernel of $\mat{Q}_x$
and therefore must add that part back using the projection onto the
kernel of $\mat{Q}_x$
\[
  \mat{P}^\text{roj}_x = \mat{I} - \mat{Q}_x \mat{Q}_x^T,
\]
with the same statement being true for $y$.
We therefore have
\[
  y = \mat{H} \mu + \varepsilon
\]
\[
  \mat{Q}_y \mat{\Lambda}_y^{1/2} \mat{\Lambda}_y^{-1/2} \mat{Q}_y^T y
  + \mat{P}^\text{roj}_y y
  = \mat{H}
  \left[
    \mat{Q}_x \mat{\Lambda}_x^{1/2} \mat{\Lambda}_x^{-1/2} \mat{Q}_x^T \mu
    + \mat{P}^\text{roj} \mu
  \right]
  + \varepsilon
\]
\[
  \mat{Q}_y \mat{\Lambda}_y^{1/2} \mat{\Lambda}_y^{-1/2} \mat{Q}_y^T y
  + \mat{P}^\text{roj}_y y
  = \mat{H}
  \left[
    \mat{Q}_{lx} \mat{\Lambda}_{lx}^{1/2} \mat{\Lambda}_{lx}^{-1/2}
    \mat{Q}_{lx}^T \mu
    + \mat{Q}_{sx} \mat{\Lambda}_{sx}^{1/2} \mat{\Lambda}_{sx}^{-1/2}
    \mat{Q}_{sx}^T \mu
    + \mat{P}^\text{roj} \mu
  \right]
  + \varepsilon
\]
where
\begin{equation}\label{eq:eig decomp decomp}
  \mat{Q}_x \mat{\Lambda}_x \mat{Q}_x^T
  = \mat{Q}_{lx} \mat{\Lambda}_{lx} \mat{Q}_{lx}^T
  + \mat{Q}_{sx} \mat{\Lambda}_{sx} \mat{Q}_{sx}^T,
\end{equation}
and $\mat{Q}_{lx}$ and $\mat{Q}_{sx}$ contain the eigenvectors that
contain information about the long and short scales
respectively.
Setting $y' = \mat{\Lambda}_y^{-1/2} \mat{Q}_y^T y$, $\mu_l' =
\mat{\Lambda}_{xl}^{-1/2} \mat{Q}_{lx}^T \mu$, and $\mu_s' =
\mat{\Lambda}_{xs}^{1/2} \mat{Q}_{sx} \mu$ we have
\[
  \mat{Q}_y \mat{\Lambda}_y^{1/2} y'
  + \mat{P}^\text{roj}_y y
  = \mat{H} \mat{Q}_{lx} \mat{\Lambda}_{lx}^{1/2} \mu_l'
  + \mat{H} \mat{Q}_{sx} \mat{\Lambda}_{sx}^{1/2} \mu_s'
  + \mat{H} \mat{P}^{\text{proj}}_x \mu
  + \varepsilon.
\]
Furthermore, because $\mat{P}^\text{roj}_y y$ is orthogonal to the
columns of $\mat{Q}_y$, setting $\varepsilon' = \mat{\Lambda}_y^{-1/2}
\mat{Q}_y^T$ we have
\begin{equation}\label{eq:prime decomp}
  y'
  = \mat{\Lambda}_y^{-1/2} \mat{Q}_y^T
  \mat{H} \mat{Q}_{lx} \mat{\Lambda}_{lx}^{1/2} \mu_l'
  + \mat{\Lambda}_y^{-1/2} \mat{Q}_y^T
  \mat{H} \mat{Q}_{sx} \mat{\Lambda}_{sx}^{1/2} \mu_s'
  + \mat{\Lambda}_y^{-1/2} \mat{Q}_y^T
  \mat{H} \mat{P}^{\text{proj}}_x \mu
  + \varepsilon'.
\end{equation}
In this form, we cannot perform data assimilation on $\mu_l'$ or
$\mu_s'$ individually because they each are using the single $y'$
observation.
The question then becomes: Can we decompose $y'$ into observations of
only long and short scales as defined though $\mat{Q}_{xl}$ and
$\mat{Q}_{xs}$?

We can do this by taking the singular value decompositions of the
matrices multiplying our $\mu$'s in Eq.~\ref{eq:prime decomp}
\[
  y'
  = \mat{U}_l \mat{\Sigma}_l \mat{V}_l^T \mu_l'
  + \mat{U}_s \mat{\Sigma}_s \mat{V}_s^T \mu_s'
  + \mat{U}_{\bot} \mat{\Sigma}_{\bot} \mat{V}_{\bot}^T \mu
  + \varepsilon'
\]
If we then assume that $\mat{U}_l$, $\mat{U}_s$, and $\mat{U}_{\bot}}$
are all orthogonal to each other, we can then decompose $y'$
\[
  \mat{U}_l^T y'
  = \mat{\Sigma}_l \mat{V}_l^T \mu_l'
  + \mat{U}_l^T \varepsilon'
\]
\[
  \mat{U}_s^T y'
  = \mat{\Sigma}_s \mat{V}_s^T \mu_s'
  + \mat{U}_s^T \varepsilon'
\]
This however will only be true it the long and short scales as defined
by the eigenvectors of $\mat{P}$ can truly be assimilated separately
when taking $\mat{R}$ and $\mat{H}$ into account.


% If we have any hope of this, we first require that the first three
% terms on the left hand side of Eq.~(\ref{eq:prime decomp}) be
% orthogonal.
% To check this, we first reformulate Eq.~(\ref{eq:eig decomp decomp}):
% \begin{equation}\label{eq:eig decomp decomp full}
%   \mat{Q}_x \mat{\Lambda}_x \mat{Q}_x^T
%   = \tilde{\mat{Q}}_{x} \tilde{\mat{\Lambda}}_{lx} \tilde{\mat{Q}}_{x}^T
%   + \tilde{\mat{Q}}_{x} \tilde{\mat{\Lambda}}_{sx} \tilde{\mat{Q}}_{x}^T,
% \end{equation}
% where $\tilde{\mat{\Lambda}}_{lx}$ has zeros on the diagonal that
% correspond to short scale eigenvectors,
% $\tilde{\mat{\Lambda}}_{sx}$ has zeros on the diagonal that correspond
% to long scale eigenvectors, and both have zeros on the diagonal that
% correspond to eigenvectors that are in the kernel of $\mat{P}$.
% If we have $\tilde{\mu}_l = \tilde{\Lambda}_{lx}^{-1/2}
% \tilde{\mat{Q}}_x^T \mu$ and $\tilde{\mu}_s =
% \tilde{\Lambda}_{sx}^{-1/2} \tilde{\mat{Q}}_x^T \mu$ then we can see
% that
% \[
%   \mat{\Lambda}_y^{-1/2} \mat{Q}_y^T
%   \mat{H} \mat{Q}_{lx} \mat{\Lambda}_{lx}^{1/2} \mu_l'
%   =\mat{\Lambda}_y^{-1/2} \mat{Q}_y^T
%   \mat{H} \tilde{\mat{Q}}_{x} \tilde{\mat{\Lambda}}_{lx}^{1/2}
%   \tilde{\mu}_l
% \]
% \[\text{and}\]
% \[
%   \mat{\Lambda}_y^{-1/2} \mat{Q}_y^T
%   \mat{H} \mat{Q}_{sx} \mat{\Lambda}_{sx}^{1/2} \mu_l'
%   =\mat{\Lambda}_y^{-1/2} \mat{Q}_y^T
%   \mat{H} \tilde{\mat{Q}}_{x} \tilde{\mat{\Lambda}}_{sx}^{1/2}
%   \tilde{\mu}_s.
% \]
% We also have
% \begin{equation}\label{eq:orth mus}
%   \tilde{y}_l^T \tilde{y}_s
%   = \tilde{\mu}_l^T \tilde{\mat{\Lambda}}_{lx}^{1/2}
%   \tilde{\mat{Q}}_x^T
%   \mat{H}^T \mat{Q}_y \mat{\Lambda}_y^{-1/2}
%   \mat{\Lambda}_y^{-1/2} \mat{Q}_y^T
%   \mat{H} \tilde{\mat{Q}}_{x} \tilde{\mat{\Lambda}}_{sx}^{1/2}
%   \tilde{\mu}_s
%   = 0,
% \end{equation}
% because of the structures of $\tilde{\mat{\Lambda}}_{xl}$ and
% $\tilde{\mat{\Lambda}}_{xs}$.





\end{document}

%%% Local Variables:
%%% mode: latex
%%% TeX-master: t
%%% End:
